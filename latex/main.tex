\documentclass{article}
\usepackage[a4paper,margin=2.2cm]{geometry}
\usepackage{graphicx}
\usepackage{amsmath}
\usepackage{subcaption}    % Para subfiguras (si usas subfigure)
\usepackage{float} 
\usepackage{amsfonts,amssymb}
\usepackage[labelfont=bf]{caption}

\title{Procesamiento de Imágenes mediante Variación Total}
\author{Joaquín Mir Macías, Miguel Montes Lorenzo, Manuel Rodríguez Villegas}
\date{October 2025}

\begin{document}

\maketitle

\section{Introducción}

\section{Introducción}

El cálculo de variaciones constituye una de las herramientas fundamentales en
matemáticas aplicadas cuando se desea optimizar una cantidad que depende de una
función desconocida. A diferencia del cálculo diferencial clásico, centrado en
maximizar o minimizar funciones de unas pocas variables reales, en cálculo
variacional la incógnita es una función completa y el objetivo consiste en
estudiar un funcional que asigna un valor real a cada función admisible. Este
enfoque permite modelar de forma natural sistemas físicos, geométricos o de
procesado de señal cuyo comportamiento global viene descrito mediante
integrales de energía.\\

Un ejemplo histórico que ilustra esta idea es el problema de la
braquistócrona, planteado por Johann Bernoulli en 1696. El problema consiste en
determinar qué curva recorre una partícula sometida únicamente a la gravedad
cuando desciende entre dos puntos en el menor tiempo posible. Aunque la
intuición podría sugerir que la solución es un segmento recto, el análisis
variacional conduce a una ecuación diferencial cuya solución es una cicloide.
Esta formulación motivó el desarrollo sistemático del cálculo de variaciones y
de herramientas fundamentales como la ecuación de Euler-Lagrange y la forma de
Beltrami, que siguen siendo pilares de la teoría moderna.\\

En las últimas décadas, estas mismas ideas se han vuelto especialmente
relevantes en el procesamiento de imágenes. Muchas tareas clásicas, como la
reducción de ruido, la recuperación de contornos, la segmentación o el
\emph{inpainting}, pueden expresarse mediante funcionales que equilibran dos
efectos complementarios, la fidelidad a los datos observados y la regularidad o
suavidad que se desea imponer sobre la imagen restaurada. En este contexto, la
variación total (TV) ha demostrado ser un regularizador especialmente eficaz,
ya que elimina ruido preservando discontinuidades relevantes, algo que no ocurre
con modelos cuadráticos más tradicionales.\\

El modelo de Rudin, Osher y Fatemi (ROF), propuesto en 1992, es uno de los
ejemplos más influyentes de este enfoque. El modelo busca una imagen restaurada
mediante la minimización de un funcional que combina la variación total con un
término de fidelidad en norma $L^2$. La presencia del término TV permite
preservar bordes nítidos, mientras que el parámetro de regularización controla
el equilibrio entre suavizado y fidelidad a la imagen original. A pesar de su
formulación sencilla, el impacto del modelo ROF ha sido notable en teoría y en
aplicaciones, y ha dado lugar a numerosas extensiones que se utilizan hoy en
día en visión por computador y procesamiento de imágenes médicas.\\

El objetivo de este trabajo es mostrar cómo las herramientas del cálculo de
variaciones, introducidas a través del problema de la braquistócrona, conducen
de manera natural a modelos modernos de restauración de imágenes. Para ello,
comenzamos revisando el planteamiento del problema clásico y la deducción de la
ecuación de Euler--Lagrange. A continuación, presentamos el modelo ROF y
analizamos su interpretación geométrica, sus propiedades fundamentales y el
efecto de sus parámetros. Finalmente, aplicamos diferentes métodos basados en
variación total, tanto el modelo clásico como extensiones recientes, a imágenes
sintéticas y médicas, lo que permite comparar su comportamiento y discutir
ventajas e inconvenientes de cada enfoque.\\


\section{Problema de la Braquistócrona}
La formulación y posterior resolución del problema de la braquistócrona supusieron el comienzo del cálculo de variaciones, donde se busca encontrar extremos relativos de funcionales continuos definidos sobre algún espacio funcional.\\

Dados los puntos A y B en el plano vertical, el problema de la braquistócrona consiste en encontrar la curva que sigue una partícula que se desplaza de A a B en el mínimo tiempo posible, únicamente bajo el efecto de la gravedad.\\

\begin{figure}[htbp]
    \centering
    \includegraphics[width=0.6\textwidth]{braquis.pdf}
    \caption{Posibles soluciones al problema de la braquistócrona.}
    \label{fig:img_braquis}
\end{figure}

Fue planteado por Johann Bernoulli en 1696 y resuelto por Newton en 1697, aunque otros cinco matemáticos (Johann y Jakob Bernoulli, Leibniz, L'Hôpital y Tschirnhaus) también intervinieron en la solución.\\

\subsection*{Planteamiento}

Para hallar la solución, tenemos que minimizar el tiempo empleado por la partícula. Para ello, representamos el tiempo en función del espacio recorrido y la velocidad de la partícula.\\

Si la curva que describe la trayectoria se expresa como $y=y(x)$, la longitud de arco entre $x_a$ y $x_b$ es
\[
s=\int_{x_a}^{x_b}\sqrt{1+\bigl(y'(x)\bigr)^2}\,dx,
\]
con la notación $ y'=\dfrac{dy}{dx} $.\\

Partiendo del principio de la conservación de la energía, sabemos que la energía cinética y la energía potencial deben ser iguales en todo momento. Por tanto, si $y$ es la altura vertical medida desde el punto de partida, la energía del sistema viene dada por
\[
\frac{1}{2} m v^{2} = m g y \quad\Longrightarrow\quad v=\sqrt{2 g y},
\]
donde $g$ es la aceleración de la gravedad y $v$ la velocidad del cuerpo en el punto considerado.\\

% Parte del código donde defines el funcional tiempo
El tiempo que tarda la partícula en desplazarse entre dos puntos viene dado por 
$t = \dfrac{s}{v}$. Para utilizar la expresión integral del espacio recorrido, podemos expresar 
el tiempo diferencial como $ dt = \dfrac{ds}{v} $ y obtener el tiempo total integrando. 
Sustituyendo $ds$ y $v$ obtenemos el funcional tiempo:
\[
t=\mathcal{T}[y]=\int_{s_a}^{s_b}\frac{ds}{v}
=\int_{x_a}^{x_b}\frac{\sqrt{1+y'(x)^{2}}}{\sqrt{2g}\,\sqrt{y(x)}}\,dx
= \int_{x_a}^{x_b} f\bigl(y(x),y'(x)\bigr)\,dx,
\]
donde hemos definido
\[
f(y,y'):=\frac{\sqrt{1+y'^{2}}}{\sqrt{2gy}}.
\]

% Aplicamos Euler-Lagrange
En este punto aplicamos la ecuación de Euler--Lagrange:
\[
\frac{\partial f}{\partial y}-\frac{d}{dx}\!\left(\frac{\partial f}{\partial y'}\right)=0.
\]

Como la función \( f \) no depende explícitamente de \( x \), entonces:
\[
\frac{df}{dx} = \frac{\partial f}{\partial y'} \frac{dy'}{dx} = 0
\]

Luego, podemos multiplicar \( \dfrac{dy}{dx} \) a la ecuación de Euler–Lagrange y restarle la expresión anterior sin modificar nada. Así, nos queda:
\[
\left(
\frac{\partial f}{\partial y}
- \frac{d}{dx} \left( \frac{\partial f}{\partial y'} \right)
\right) \frac{dy}{dx}
- \frac{\partial f}{\partial y'} \frac{dy'}{dx}
= 0
\]

Desarrollando los términos:
\[
\frac{\partial f}{\partial y} \frac{dy}{dx}
- \frac{d}{dx} \left( \frac{\partial f}{\partial y'} \right) \frac{dy}{dx}
- \frac{\partial f}{\partial y'} \frac{dy'}{dx}
= 0
\]

En este punto podemos simplificar el primer sumando y agrupar los otros dos mediante una derivación:
\[
\frac{df}{dx} - \frac{d}{dx} \left( y' \frac{\partial f}{\partial y'} \right) = 0
\]

resultando en una expresión conocida como el primer integral o forma de Beltrami.\\

Así, al integrar vemos que nos aparece un término constante:

\[
\frac{d}{dx}\left(f-y'{\frac {\partial f}{\partial y'}}\right)=0
\quad\Longrightarrow\quad
f - y'\frac{\partial f}{\partial y'} = C.
\]

% Cálculo explícito de derivadas
Calculamos las derivadas necesarias:
\[
f(y,y')=\frac{\sqrt{1+y'^{2}}}{\sqrt{2gy}},\quad
\frac{\partial f}{\partial y'}=\frac{y'}{\sqrt{2gy}\,\sqrt{1+y'^{2}}}
\]

Sustituyendo, llegamos a:

\[
\frac{\sqrt{1+y'^{2}}}{\sqrt{2gy}} - y' \frac{y'}{\sqrt{2gy}\,\sqrt{1+y'^{2}}}=
\frac{1}{\sqrt{2gy}\,\sqrt{1+y'^{2}}}=C,
\]

Despejando $y'$:
\[
1+y'^{2}=\frac{1}{2gC^{2}y}\quad\Longrightarrow\quad
\frac{dy}{dx}=y'=\pm\sqrt{\frac{1}{2gC^{2}y}-1}.
\]

% Solución paramétrica cicloide
\subsection*{Solución paramétrica: cicloide}
La solución que cumple $y(0)=0$ puede expresarse mediante parámetros angulares utilizando la expresión de la cicloide:

\[
\begin{cases}
x(\theta)=\dfrac{1}{4gC^{2}}\bigl(\theta-\sin\theta\bigr),\\[2mm]
y(\theta)=\dfrac{1}{4gC^{2}}\bigl(1-\cos\theta\bigr),
\end{cases}
\qquad \theta\in[0,\theta_f].
\]

\begin{figure}[htbp]
    \centering
    \includegraphics[width=0.6\textwidth]{ciclo.pdf}
    \caption{Construcción de la curva cicloide.}
    \label{fig:img_ciclo}
\end{figure}

Verificamos que esta curva satisface la ecuación para $y'$. Calculando derivadas respecto de $\theta$:
\[
\frac{dx}{d\theta}=\frac{1}{4gC^{2}}\bigl(1-\cos\theta\bigr),\qquad
\frac{dy}{d\theta}=\frac{1}{4gC^{2}}\sin\theta,
\]

por lo que
\[
\frac{dy}{dx}=\frac{\dfrac{dy}{d\theta}}{\dfrac{dx}{d\theta}}
= \frac{\sin\theta}{1-\cos\theta}= \sqrt{\frac{1+\cos\theta}{1-\cos\theta}}
\]


Si sustituimos

\[
y(\theta)=\dfrac{1}{4gC^{2}}(1-\cos\theta)
\]

en la expresión

\[
\sqrt{\frac{1}{2gC^{2}y}-1},
\]

nos queda:

\[
\frac{1}{2gC^{2}y}
= \frac{1}{2gC^{2}\cdot \dfrac{1-\cos\theta}{4gC^{2}}}
= \frac{1}{\dfrac{1-\cos\theta}{2}}
= \frac{2}{1-\cos\theta}
\]

Restando 1:

\[
\frac{1}{2gC^{2}y}-1
= \frac{2}{1-\cos\theta}-1
= \frac{2-(1-\cos\theta)}{1-\cos\theta}
= \frac{1+\cos\theta}{1-\cos\theta}.
\]

Haciendo la raíz cuadrada:

\[
\sqrt{\frac{1}{2gC^{2}y}-1}
= \sqrt{\frac{1+\cos\theta}{1-\cos\theta}}.
\]

Por lo tanto

\[
\frac{dy}{dx}=\frac{\sin\theta}{1-\cos\theta}
= \sqrt{\frac{1+\cos\theta}{1-\cos\theta}}
= \sqrt{\frac{1}{2gC^{2}y}-1},
\]

y la parametrización por la cicloide satisface la condición impuesta por el primer integral.



\section{Procesamiento de Imágenes}
Una vez analizado el desarrollo teórico de un problema de cálculo variacional, profundicemos en una de las múltiples aplicaciones en la vida real: el procesamiento de imágenes.
\subsection{Modelo de Rudin--Osher--Fatemi}

El modelo de Rudin--Osher--Fatemi (ROF), propuesto en 1992 , representa uno de los hitos fundamentales en el procesamiento de imágenes basado en variación total. Su objetivo es eliminar ruido preservando los bordes y las estructuras relevantes de la imagen. 

Dado un dominio $\Omega\subset\mathbb{R}^2$ y una imagen ruidosa $f:\Omega\to\mathbb{R}$, se busca una imagen restaurada $u$ que minimice la energía total del sistema bajo el principio de equilibrio entre fidelidad a los datos y suavidad de la imagen. 

El modelo puede formularse como el problema de minimización:
\[
\min_u \int_\Omega |\nabla u| \,dx \quad \text{sujeto a} \quad 
\int_\Omega u = \int_\Omega f,\qquad 
\int_\Omega |f - u|^2 = \sigma^2,
\]
donde $\sigma^2$ es la varianza global del ruido aditivo.

Una formulación equivalente con restricción suave es
\[
u_\lambda = \arg\min_u \left\{ |u|_{BV(\Omega)} + 
\lambda \|f - u\|^2_{L^2(\Omega)} \right\},
\]
donde $|u|_{BV(\Omega)}$ es el seminorma de variación total y $\lambda>0$ regula el compromiso entre suavizado y fidelidad. 

En general, este problema pertenece a la familia de funcionales convexos
\[
u_\lambda = \arg\min_u \{ J(u) + H(f,u) \},
\]
donde $J$ es un funcional de regularización no negativo y convexo, y $H$ es un funcional convexo de ajuste a datos. Por convexidad, existe un minimizador $u_\lambda$ tal que 
\[
0 \in \partial_u J(u_\lambda) + \partial_u H(f,u_\lambda),
\]
siendo $\partial_u$ el subdiferencial respecto a $u$. 

Para el caso particular $H(f,u) = \lambda \|f - u\|^2_{L^2}$, se tiene $\partial_u H = 2\lambda(u-f)$. Además, el subdiferencial del término de variación total puede formalmente escribirse como
\[
\partial J(u) = -\nabla\cdot\left(\frac{\nabla u}{|\nabla u|}\right),
\]
por lo que la ecuación de Euler–Lagrange del modelo ROF toma la forma
\[
0 = -\nabla\cdot\!\left(\frac{\nabla u}{|\nabla u|}\right) + 2\lambda(u-f),
\]
o equivalentemente,
\[
u = f + \frac{1}{2\lambda}\nabla\cdot\!\left(\frac{\nabla u}{|\nabla u|}\right).
\]

Se impone la condición de contorno de Neumann $\nabla u\cdot\nu=0$ sobre $\partial\Omega$, que garantiza la conservación de la media: $\int_\Omega u = \int_\Omega f$.

\subsubsection*{Influencia del parámetro $\lambda$}

El parámetro $\lambda$ determina el nivel de suavizado:
\begin{itemize}
\item Para $\lambda$ pequeño, el término de fidelidad domina y $u_\lambda \approx f$, produciendo un filtrado leve.
\item Para $\lambda$ grande, predomina la variación total, eliminando detalles finos y produciendo una imagen tipo “cartoon” que conserva las estructuras principales.
\end{itemize}

\subsubsection*{Propiedades teóricas del modelo}

\begin{enumerate}
\item \textbf{Acotación del minimizador:} La norma $L^2$ de $u_\lambda$ está acotada independientemente de $\lambda$. Dado que $u_\lambda$ minimiza el funcional, para cualquier $w \in BV(\Omega)$ se cumple
\[
|u_\lambda|_{BV(\Omega)} + \lambda\|f - u_\lambda\|^2_{L^2(\Omega)}
\le |w|_{BV(\Omega)} + \lambda\|f - w\|^2_{L^2(\Omega)}.
\]
Tomando $w\equiv 0$ se obtiene
\[
\|f - u_\lambda\|^2_{L^2(\Omega)} \le \|f\|^2_{L^2(\Omega)},
\]
y mediante desigualdad de Cauchy–Schwarz se concluye que $\|u_\lambda\|_{L^2(\Omega)} \le 2\|f\|_{L^2(\Omega)}$.

\item \textbf{Conservación de la media:} Integrando la ecuación de Euler–Lagrange y aplicando la condición de Neumann, se cumple que
\[
\int_\Omega u_\lambda = \int_\Omega f.
\]
\end{enumerate}

\subsubsection*{Ejemplo práctico: aplicación sobre la imagen de Lena}

Consideremos una imagen $f$ en escala de grises con rango de valores en $[0,255]$ contaminada con ruido gaussiano aditivo. A continuación se muestran los resultados cualitativos del modelo ROF con distintos valores de $\lambda$:

\begin{itemize}
\item $\lambda=0.8$: la imagen $u_\lambda$ es muy similar a $f$; la eliminación de ruido es leve.
\item $\lambda=0.05$: el ruido se reduce significativamente, manteniendo contornos y estructuras esenciales.
\item $\lambda=0.01$: los detalles finos desaparecen, obteniendo una imagen \emph{cartoonizada}, donde sólo permanecen las estructuras globales.
\end{itemize}

\begin{figure}[h!]
    \centering
    \includegraphics[width=0.45\textwidth]{lena_original.png}
    \includegraphics[width=0.45\textwidth]{lena_denoised_lambda_0_05.png}
    \caption{Resultado del modelo ROF aplicado a la imagen \emph{Lena}. Izquierda: imagen original ruidosa $f$. Derecha: resultado $u_\lambda$ para $\lambda=0.05$.}
\end{figure}

\subsubsection*{Interpretación geométrica y extensiones}

La minimización del término $\int_\Omega |\nabla u|$ favorece soluciones con regiones homogéneas separadas por discontinuidades nítidas. Así, el modelo ROF puede entenderse como un \emph{flujo de curvatura media} regularizado, que difunde en zonas planas y se detiene cerca de bordes fuertes. 

Extensiones comunes incluyen:
\begin{itemize}
\item Modelos anisotrópicos, donde se pondera la difusión en direcciones preferentes.
\item ROF con fidelidad $L^1$, más robusto ante ruido impulsivo.
\item Modelos de segundo orden (TGV, Total Generalized Variation) que reducen el efecto de “escalonado” (\emph{staircasing}).
\end{itemize}

\subsubsection*{Simulación numérica}

El modelo puede resolverse iterativamente mediante un flujo de gradiente descendente:
\[
\partial_t u = f - u + \lambda\,\nabla\cdot\!\left(\frac{\nabla u}{|\nabla u|_\varepsilon}\right),
\quad u(\cdot,0)=f.
\]
En discretización finita, este flujo produce una secuencia $u^k$ que converge hacia el minimizador $u_\lambda$. En la práctica, se utiliza un esquema explícito o semiexplícito de diferencias finitas o bien el método de proyección dual de Chambolle, ampliamente adoptado por su estabilidad y rapidez.

\subsubsection*{Conclusión}

El modelo de Rudin--Osher--Fatemi unifica de manera elegante el cálculo variacional y la teoría del espacio de funciones de variación acotada (BV) en el contexto del procesamiento digital de imágenes. Su capacidad para reducir ruido manteniendo estructuras esenciales lo convierte en la base de muchos métodos modernos de restauración e incluso en redes neuronales variacionales actuales.

\section{Ejemplo Aplicado}

\section{Aplicación en Imágenes Médicas}
A continuación se presenta una de las aplicaciones de mayor interés del procesamiento de imágenes: reducción de ruido en imágenes médicas. En muchas ocasiones, los escáneres médicos presentan un alto nivel de ruido debido al difícil proceso de otención de dichos datos, normalmente a través de rayos-X. Por esta razón, la eliminación de ruido se convierte en un paso fundamental previo al análisis por parte de un doctor o de un sistema de visión por ordenador \cite{tonolini2020variationalinferencecomputationalimaging}.\\

En esta sección aplicamos el método de reducción de ruido explicado previamente sobre escáneres del cerebro y de los pulmones. Además, comparamos tres variantes y discutimos en qué casos es más conveniente utilizar una u otra.

\subsection{Método Clásico -- Rudin-Osher-Fatemi}
En primera instancia empleamos el método Rudin-Osher-Fatemi original sobre escáneres médicos. En función del parámetro $\lambda$ obtendremos un resultado con más ruido pero bordes mejor conservados, o un ruido más reducido a costa de perder algo de definición.

Así, distinguimos tres casos en los que el parámetro $\lambda$ óptimo puede ser diferente.\\

\begin{figure}[htbp]
    \centering
    \includegraphics[width=\textwidth]{tv_sweep_MRI_Brain_1.pdf}
    \caption{Escáner cerebral.}
    \label{fig:img1}
\end{figure}

En el escáner cerebral, donde hay una clara distinción entre tejidos (materia gris y materia blanca), observamos con claridad cómo aumentar $\lambda$ reduce mucho la definición de la imagen, por lo que es preferible utilizar un valor más bajo que, pese a mantener algo más de ruido, ofrece un resultado con mayor resolución. Las imágenes de escáneres cerebrales han sido optenidas del paquete \texttt{skimage} de Python \cite{vandervwalt2014scikitimage}.\\

\begin{figure}[htbp]
    \centering
    \includegraphics[width=\textwidth]{tv_sweep_Xray_Bacteria_1.pdf}
    \caption{Escáner pulmonar en presencia de una bacteria.}
    \label{fig:img2}
\end{figure}

En el primer escáner pulmonar, observamos cómo un $\lambda$ más pequeño tiene una influencia casi nula en la eliminación de ruido, mientras que valores superiores sí consiguen eliminarlo más satisfactoriamente. En este caso, la pérdida de resolución puede compensar la superior eliminación de ruido, llegando a obtener una imagen casi idéntica a la original.\\

\begin{figure}[htbp]
    \centering
    \includegraphics[width=\textwidth]{tv_sweep_Xray_Virus_1.pdf}
    \caption{Escáner pulmonar en presencia de un virus.}
    \label{fig:img3}
\end{figure}

Por último, observando la tercera imagen, percibimos un efecto similar al anterior, y es que en este tipo de escáneres suele ser preferible perder algo de resolución para obtener una imagen casi completamente libre de ruido. Los escáneres pulmonares han sido obtenidos del dataset público de Kaggle Chest X-Ray Images (Pneumonia) \cite{mooney2018chestxray}.
\subsection{Método con Adaptación Espacial}

Para reducir ruido sin degradar contornos finos en imágenes médicas, consideramos
una extensión del modelo ROF con \emph{ponderación espacial}. Definimos el funcional
\[
\min_{u}\; \Bigg\{
\underbrace{\int_\Omega w(x)\,|\nabla u(x)|\,dx}_{\text{TV ponderada}}
\;+\;
\underbrace{\frac{1}{2}\int_\Omega \lambda(x)\,\bigl(u(x)-f(x)\bigr)^2\,dx}_{\text{fidelidad espacial}}
\Bigg\},
\]
donde $w:\Omega\to[0,1]$ atenúa la regularización cerca de bordes y
$\lambda:\Omega\to\mathbb{R}_+$ permite ajustar localmente la fidelidad a datos.
La elección típica es $\lambda(x)\equiv\lambda>0$ constante y un $w$ dependiente de
un \emph{indicador de borde} estable:
\[
w(x)=\exp\!\left(-\Bigl(\tfrac{|\nabla (G_\sigma * f)(x)|}{k}\Bigr)^\beta\right),
\qquad \beta\in[1,4],
\]
donde $G_\sigma$ es un filtro gaussiano (robustez), $k>0$ escala el umbral de borde
(p.\,ej.\ $k$ tomado como un percentil del módulo de gradiente) y
$|\nabla (G_\sigma * f)|$ es el módulo del gradiente de la imagen suavizada.

\paragraph{Ecuación de Euler–Lagrange.}
Para $u$ suficientemente regular y $|\nabla u|\not=0$ a.e.,
el subgradiente de la TV ponderada es
\[
\partial\!\left(\int w\,|\nabla u|\right)
= -\,\nabla\cdot\!\left(w\,\frac{\nabla u}{|\nabla u|}\right).
\]
La condición de estacionariedad del funcional anterior con condiciones de Neumann
($\nabla u\cdot\nu=0$ en $\partial\Omega$) conduce a
\[
0 \;=\; -\,\nabla\cdot\!\left(w\,\frac{\nabla u}{|\nabla u|}\right)
\;+\;\lambda(x)\,(u-f),
\]
o de forma explícita,
\begin{equation}
\boxed{%
\lambda(x)\,(u-f)\;-\;\nabla\cdot\!\left(w(x)\,\frac{\nabla u}{|\nabla u|}\right)=0.}
\label{eq:EL_adapt}
\end{equation}

\paragraph{Esquema numérico.}
Resolvemos \eqref{eq:EL_adapt} por descenso de gradiente en tiempo artificial:
\[
\partial_t u
\;=\; \nabla\cdot\!\left(w\,\frac{\nabla u}{\sqrt{|\nabla u|^2+\varepsilon^2}}\right)
\;+\;\lambda(x)\,(f-u),
\quad u(\cdot,0)=f,
\]
donde $\varepsilon\ll 1$ estabiliza la norma y preserva bordes fuertes.
Discretizamos con diferencias finitas y condiciones de Neumann
(extension por borde) usando un paso temporal $\,\Delta t\in[0.1,0.25]$ y un número
de iteraciones fijo (p.\,ej.\ 200--300) hasta convergencia visual.

\paragraph{Construcción de $w$.}
Calculamos $g_\sigma = G_\sigma * f$, su gradiente por diferencias finitas y el
módulo $|\nabla g_\sigma|$. Tomamos $k$ como el percentil $p$ (p.\,ej.\ $p=90$)
de $|\nabla g_\sigma|$ para adaptar el umbral a cada imagen, y fijamos $\beta\in[1,4]$.
Así, $w\approx 1$ en zonas casi planas (más suavizado) y $w\approx 0$ a lo largo
de contornos (suavizado inhibido), lo cual mantiene estructuras anatómicas clave.

\paragraph{Observaciones prácticas.}
\begin{itemize}
\item En imágenes de RM cerebrales, $w(x)$ preserva interfaces materia blanca/gris;
$\lambda$ moderado evita sobrealisado de textura clínica relevante.
\item En radiografías de tórax, la atenuación en fisuras y bordes pulmonares
permite valores de $\lambda$ más altos para limpiar ruido sin perder anatomía fina.
\item La elección de $(\sigma, p, \beta)$ gobierna la \emph{selectividad} de bordes.
\end{itemize}

\subsection{Método de Orden Superior (Regularizadores)}

Por último, estudiamos un método de denoising que introduce regularización de orden superior. 
La idea es sencilla: en lugar de penalizar únicamente el gradiente de la imagen, como ocurre en ROF, 
añadimos también un término que penaliza el laplaciano. De esta forma, el modelo tiende a producir 
transiciones más suaves y evita el efecto de “staircasing”, que aparece con bastante frecuencia cuando 
se usa variación total clásica.

El funcional que queremos minimizar es
\[
J[u]
=
\frac{1}{2}\int_\Omega (u-f)^2\,dx
+
\frac{\alpha}{2}\int_\Omega |\nabla u|^2\,dx
+
\frac{\beta}{2}\int_\Omega (\Delta u)^2\,dx ,
\]
donde $f$ es la imagen ruidosa y los parámetros $\alpha$ y $\beta$ controlan la intensidad del suavizado.
El término con $\alpha$ se comporta como una regularización estándar de primer orden mientras que $\beta$
controla el efecto de segundo orden, que es el que permite eliminar cambios bruscos sin generar regiones
por tramos constantes.

Derivando este funcional se obtiene la ecuación de Euler--Lagrange asociada:
\[
0 = (u-f) - \alpha \Delta u + \beta\,\Delta^{2} u,
\]
donde $\Delta^2 u$ es el bilaplaciano. Resolver esta ecuación de forma directa no es sencillo, así que 
utilizamos un descenso de gradiente explícito:
\[
\partial_t u = (f-u) + \alpha\,\Delta u - \beta\,\Delta^2 u,
\qquad u(\cdot,0) = f.
\]
Para implementar este método utilizamos diferencias finitas, tanto para el laplaciano como para el bilaplaciano (que calculamos aplicando dos veces el laplaciano). Las condiciones de contorno son de Neumann para mantener la imagen dentro del rango esperado. En la práctica, basta elegir un paso de tiempo pequeño y realizar varias iteraciones hasta que los cambios entre pasos sean mínimos.\\

Desde el punto de vista práctico, el comportamiento del método es algo mejor que el de los dos previos. El regularizador de orden superior genera imágenes mucho más suaves y sin artefactos, y suele eliminar el ruido de forma bastante uniforme. Como consecuencia, los bordes se vuelven algo menos nítidos, aunque 
sin perder completamente la estructura general. Esto se nota sobre todo en radiografías, donde los detalles finos desaparecen antes que en el caso del TV adaptativo. En cambio, en escáneres cerebrales, que tienen regiones amplias de intensidad casi constante, el método funciona especialmente bien porque reduce el ruido sin introducir patrones artificiales.\\

Como ejemplo, en la Figura~\ref{fig:imgHO1} mostramos una comparación del método con distintos valores de 
$\alpha$ y $\beta$. En todos los casos se observa un suavizado más homogéneo respecto al modelo ROF, aunque 
a costa de perder algo de detalle en los bordes.

\begin{figure}[htbp]
    \centering
    \includegraphics[width=\textwidth]{images/higher_order_sweep_MRI_Brain_1.pdf}
    \caption{Resultados del regularizador de orden superior en un escáner cerebral.}
    \label{fig:imgHO1}
\end{figure}



\section*{1. Idea general y motivación}

Dada una imagen ruidosa \( f(x,y) \), se desea obtener una versión más suave \( u(x,y) \) que conserve una gran similitud con la imagen original pero que elimine gran parte del ruido presente.
El modelo propuesto por Rudin, Osher y Fatemi \cite{Rudin1992} (1992) se basa en minimizar el siguiente funcional

\begin{equation}
J[u] = 
\frac{1}{2} \int_{\Omega} (u - f)^2 \, dx\,dy
+ \lambda \int_{\Omega} |\nabla u| \, dx\,dy,
\label{eq:rof}
\end{equation}

donde el primer término mide la fidelidad a los datos y el segundo, la variación total (TV), actúa como término de regularización.

El término de fidelidad ayuda a que la nueva imagen permanezca cerca de la imagen original \(f\).
El término de variación total penaliza la magnitud del gradiente, controlando la rugosidad de la imagen sin castigar en exceso las discontinuidades.\\

A diferencia de la regularización cuadrática 
\(\int |\nabla u|^2\),
la penalización lineal en \(|\nabla u|\) permite la existencia de saltos finitos en \(u\). Esto provoca que se supriman oscilaciones de alta frecuencia, normalmente relacionadas con el ruido, y que se preserven los bordes significativos.\\

El parámetro \(\lambda > 0\) controla el equilibrio entre suavizado y fidelidad:
valores grandes de \(\lambda\) producen una imagen muy lisa, eliminando grande parte del ruido pero difuminando significativamente la imagen (tratando que \(|\nabla u|\) sea lo más bajo posible),
mientras que valores pequeños dejan más ruido.
Como veremos posteriormente, su ajuste óptimo depende del nivel de ruido en la imagen.\\

Este método también conlleva una serie de limitaciones, entre las que destacan la pérdida de contraste en objetos pequeños homogéneos o artefactos de \emph{staircasing} (zonas planas separadas por saltos).

\subsection*{Derivación de la condición de Euler--Lagrange}

Consideremos el funcional \eqref{eq:rof}.  
La condición de estacionariedad se obtiene imponiendo que la variación de \(J[u]\) se anule
para toda función de variación variación \(\eta\) tal que \( \eta(a) = \eta(b) = 0 \):

\[
\left.\frac{d}{d\varepsilon} J[u+\varepsilon \eta]\right|_{\varepsilon=0} = 0.
\]

El término de fidelidad queda de la siguiente manera:

\[
\frac{d}{d\varepsilon}\Big|_{\varepsilon=0}
\frac{1}{2}\int (u+\varepsilon \eta - f)^2
= \int (u-f)\,\eta \,dx\,dy.
\]

Por otro lado, suponiendo \(\nabla u \neq 0\) en la región de interés, el término de variación total queda en:

\[
\frac{d}{d\varepsilon}\Big|_{\varepsilon=0}
\int |\nabla (u+\varepsilon \eta)|
= \int \frac{\nabla u}{|\nabla u|}\cdot\nabla \eta \,dx\,dy.
\]

Aplicando integración por partes y descartando el término de frontera por las condiciones de contorno, obtenemos

\[
\int \frac{\nabla u}{|\nabla u|}\cdot\nabla \eta
= -\int \operatorname{div}\!\left(\frac{\nabla u}{|\nabla u|}\right) \eta \,dx\,dy.
\]

Sumando ambos términos y exigiendo que la variación total sea nula para toda \(\eta\), se obtiene

\[
\int \left[
(u-f)
- \lambda\, \operatorname{div}\!\left(\frac{\nabla u}{|\nabla u|}\right)
\right] \eta \, dx\,dy = 0.
\]

Por la arbitrariedad de \(\eta\), la condición de Euler--Lagrange se cumple:

\begin{equation}
\boxed{
(u - f) - \lambda\, \nabla\cdot\!\left(\frac{\nabla u}{|\nabla u|}\right) = 0.
}
\label{eq:EL}
\end{equation}

Esta es la ecuación fundamental del modelo de Rudin--Osher--Fatemi en su formulación clásica.

\subsection*{Interpretación geométrica del término de curvatura}

El vector
\[
\frac{\nabla u}{|\nabla u|}
\]
representa la dirección normal unitaria a las curvas de nivel de \(u\).
Su divergencia,
\[
\nabla\cdot\!\left(\frac{\nabla u}{|\nabla u|}\right),
\]
corresponde a la curvatura media de dichas curvas en dos dimensiones.\\

En la ecuación \eqref{eq:EL}, este término provoca un efecto de suavizado:
reduce la curvatura de contornos pequeños (eliminando ruido fino)
pero no obliga a eliminar discontinuidades grandes, lo que permite conservar bordes nítidos.\\

En zonas homogéneas, el gradiente es pequeño y la ecuación difunde la intensidad, promediando los valores y eliminando ruido. En los bordes, donde \( |\nabla u| \) es grande, el término de curvatura se atenúa y evita la difusión transversal, preservando así los contornos.

En la práctica, cuando \( |\nabla u| \) es muy pequeño, el cociente \( \nabla u / |\nabla u| \) se vuelve inestable.
Para evitar singularidades, se utiliza la regularización
\[
|\nabla u| \approx \sqrt{|\nabla u|^2 + \varepsilon^2}, \quad \varepsilon \ll 1.
\]
Esto garantiza estabilidad numérica y permite resolver \eqref{eq:EL} mediante esquemas iterativos.\\

Una forma práctica de obtener \(u\) consiste en realizar un descenso de gradiente
del funcional \(J[u]\) en un tiempo artificial \(t\):

\[
\frac{\partial u}{\partial t}
= \nabla\cdot\!\left(\frac{\nabla u}{|\nabla u|}\right)
+ \lambda (f - u).
\]

El estado estacionario (cuando \( \partial_t u = 0 \)) coincide con la ecuación de Euler--Lagrange \eqref{eq:EL}.
Esta ecuación puede interpretarse como una difusión no lineal: difunde dentro de regiones homogéneas y se detiene en bordes.





\section{Conclusiones}

A lo largo del trabajo hemos visto cómo una misma idea, minimizar un funcional mediante cálculo de variaciones, aparece tanto en un problema clásico de mecánica como la braquistócrona como en aplicaciones actuales de procesado de imagen. En ambos casos el punto de partida es muy parecido, escribir una cantidad de interés en forma de integral y buscar la función que la hace mínima utilizando la ecuación de Euler--Lagrange.\\

En la parte aplicada hemos comparado tres modelos distintos de reducción de ruido en imágenes. El modelo ROF sirve como referencia básica, es sencillo de formular y de implementar y consigue eliminar una parte importante del ruido sin destruir completamente los bordes, aunque puede introducir zonas por tramos constantes y pérdida de contraste en detalles pequeños. El modelo con adaptación espacial mejora este comportamiento en muchas imágenes médicas, ya que suaviza sobre todo en regiones homogéneas y respeta mejor las estructuras marcadas por el gradiente, lo que resulta especialmente útil en radiografías de tórax. El regularizador de orden superior produce imágenes todavía más suaves y sin artefactos de escalonado, y funciona bien en escáneres cerebrales con grandes áreas casi constantes, aunque tiende a difuminar algo más los contornos finos.\\

En resumen, ningún método es universalmente mejor que los demás, sino que la elección depende del tipo de imagen y de lo que se quiera priorizar, conservación de bordes, eliminación agresiva de ruido o ausencia de artefactos. Aun así, los tres modelos comparten el mismo marco variacional, lo que muestra la flexibilidad de estas técnicas y explica por qué siguen siendo una herramienta muy útil tanto en teoría como en aplicaciones de procesamiento de imágenes.





\bibliographystyle{plain}
\bibliography{references}

\end{document}